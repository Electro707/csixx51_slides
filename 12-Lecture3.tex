% This program is free software: you can redistribute it and/or modify it under the terms of the GNU General Public License as published by the Free Software Foundation, either version 3 of the License, or (at your option) any later version.
%
% This program is distributed in the hope that it will be useful, but WITHOUT ANY WARRANTY; without even the implied warranty of MERCHANTABILITY or FITNESS FOR A PARTICULAR PURPOSE. See the GNU General Public License for more details.
%
% You should have received a copy of the GNU General Public License along with this program. If not, see <https://www.gnu.org/licenses/>.

% This program is free software: you can redistribute it and/or modify it under the terms of the GNU General Public License as published by the Free Software Foundation, either version 3 of the License, or (at your option) any later version.
%
% This program is distributed in the hope that it will be useful, but WITHOUT ANY WARRANTY; without even the implied warranty of MERCHANTABILITY or FITNESS FOR A PARTICULAR PURPOSE. See the GNU General Public License for more details.
%
% You should have received a copy of the GNU General Public License along with this program. If not, see <https://www.gnu.org/licenses/>.

\documentclass[10pt,presentation]{beamer}

\usetheme[progressbar=frametitle, numbering=fraction, background=dark]{metropolis}
\usepackage{appendixnumberbeamer}

\usepackage{booktabs}
\usepackage[scale=2]{ccicons}

\usepackage{hyperref}
\usepackage{multimedia}

\usepackage{enumerate}
\usepackage[skip=2pt]{caption}

\usepackage{dirtytalk}
\usepackage{multimedia}
\usepackage[outputdir=build]{minted}

\usepackage{tikz}
\usepackage{tikzducks}
\usepackage{bookmark}

% https://tex.stackexchange.com/questions/152392/date-format-yyyy-mm-dd
\usepackage[yyyymmdd]{datetime}
\renewcommand{\dateseparator}{--}

% Copied from https://github.com/matze/mtheme/blob/master/source/beamerinnerthememetropolis.dtx
\setbeamertemplate{title page}{
  \begin{minipage}[b][\paperheight]{\textwidth}
    \vfill%
    \ifx\inserttitle\@empty\else\usebeamertemplate*{title}\fi
    \ifx\insertsubtitle\@empty\else\usebeamertemplate*{subtitle}\fi
    \usebeamertemplate*{title separator}
    \ifx\inserttitlegraphic\@empty\else\usebeamertemplate*{title graphic}\fi
    \ifx\beamer@shortauthor\@empty\else\usebeamertemplate*{author}\fi
    \ifx\insertdate\@empty\else\usebeamertemplate*{date}\fi
    \ifx\insertinstitute\@empty\else\usebeamertemplate*{institute}\fi
    \vfill
    {\tiny Copyright (C) 2023 Jamal Bouajjaj under GPLv3} \par
    \vspace*{5mm}
    \vspace*{1mm}
  \end{minipage}
}

\setbeamertemplate{title graphic}{
  \vbox to 0pt {
    \vspace*{0em}
    \inserttitlegraphic%
  }%
  \nointerlineskip%
}

\titlegraphic{\hfill \begin{tikzpicture}[scale=0.75]
\duck[magichat,magicwand,magicstars=blue!60!white,squareglasses=blue!50!black]
\end{tikzpicture}}

\newenvironment{Figure}
  {\par\medskip\noindent\minipage{\linewidth}\centering}
  {\endminipage\par\medskip}

\newcommand\Wider[2][3em]{%
\makebox[\linewidth][c]{%
  \begin{minipage}{\dimexpr\textwidth+#1\relax}
  \raggedright#2
  \end{minipage}%
  }%
}


\title{Lecture \#3: For Loops and Exceptions}
\date{\today}
\author{Presented by Jamal Bouajjaj}
\institute{For University of New Haven's Fall 2023 CSCIxx51 Course}

\begin{document}

\maketitle

\section{For Loops}
\begin{frame}[containsverbatim]{The Loop}
    The for loop can be used to itterate over something, such as a list
    \begin{minted}[breaklines=true,frame=single]{python}
a = [2, 4, 6, 7, 7, 9]
for i in a:
    print(a)
    \end{minted}
\end{frame}

\begin{frame}[containsverbatim]{Over a Generator}
    The for loop can also itterate over a generator (more on that later)
    \begin{minted}[breaklines=true,frame=single]{python}
for i in range(20):
    print(a)
    \end{minted}
\end{frame}

\begin{frame}[containsverbatim]{Dictionary}
    The for loop can itterate over a dictionary, returning it's keys
    \begin{minted}[breaklines=true,frame=single]{python}
a = {'resistor': 1e3, 'tollerance': 0.05, 'power': 0.125}
for i in a:
    print(a)
    \end{minted}
\end{frame}

\section{Exceptions}
\begin{frame}[containsverbatim]{Errors}
    An Exception is raised whenever Python see an issue

    \begin{minted}[breaklines=true,frame=single]{python}
a = 5 / 0       # Forbidden fruit
    \end{minted}
\end{frame}

\begin{frame}[containsverbatim]{Rasing them}
    Exceptions can also be raised

    \begin{minted}[breaklines=true,frame=single]{python}
raise UserWarning("Haha very funny!")
    \end{minted}
\end{frame}

\begin{frame}[containsverbatim]{Exceptions List}
    The list of default exceptions can be found at \url{https://docs.python.org/3/library/exceptions.html}
\end{frame}

\begin{frame}[containsverbatim]{Handling them}
    Exceptions needs to handled, or else your program will crash!

    This is done with the following block

    \begin{minted}[breaklines=true,frame=single]{python}
x = input("What to divide by (please no 0): ")
x = int(x)
try:
    a = 5 / x
except ZeroDivisionError:
    print("How could you!")
else:
    print(f"Good, result is {a}")
finally:
    print("Done!")
    \end{minted}
\end{frame}

\begin{frame}[containsverbatim]{Exception-seption}
    There can also be exceptions inside exceptions. Exception-seption

    \begin{minted}[breaklines=true,frame=single]{python}
x = input("What to divide by (please no 0): ")
x = int(x)
try:
    a = 5 / x
except ZeroDivisionError:
    print("How could you! Screw you then!")
    a = int("what?")
else:
    print(f"Good, result is {a}")
finally:
    print("Done!")
    \end{minted}
\end{frame}

\begin{frame}[containsverbatim]{Catching Them All!}
    Multiple Exceptions can be handled with the same except block, by using a Tupple

    \begin{minted}[breaklines=true,frame=single]{python}
x = input("What to divide by (please no 0): ")
try:
    x = int(x)
    a = 5 / x
# Catch both a ZeroDivisionError and ValueError the same way!
except (ZeroDivisionError, ValueError):
    print("How could you!")
else:
    print(f"Good, result is {a}")
finally:
    print("Done!")
    \end{minted}
\end{frame}

\begin{frame}{Good Practices}
    There are a couple of good practice while handling exceptions
    \begin{itemize}
        \item Have them, please.
        \item Add them around most things, especially where there can be known exceptions
        \item Don't generalize: Specify the exception
        \item ...Except around your main application as a safety check
    \end{itemize}
\end{frame}

\begin{frame}[standout]{End}
  The end
\end{frame}

\end{document}
