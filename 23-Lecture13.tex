% This program is free software: you can redistribute it and/or modify it under the terms of the GNU General Public License as published by the Free Software Foundation, either version 3 of the License, or (at your option) any later version.
%
% This program is distributed in the hope that it will be useful, but WITHOUT ANY WARRANTY; without even the implied warranty of MERCHANTABILITY or FITNESS FOR A PARTICULAR PURPOSE. See the GNU General Public License for more details.
%
% You should have received a copy of the GNU General Public License along with this program. If not, see <https://www.gnu.org/licenses/>.

% This program is free software: you can redistribute it and/or modify it under the terms of the GNU General Public License as published by the Free Software Foundation, either version 3 of the License, or (at your option) any later version.
%
% This program is distributed in the hope that it will be useful, but WITHOUT ANY WARRANTY; without even the implied warranty of MERCHANTABILITY or FITNESS FOR A PARTICULAR PURPOSE. See the GNU General Public License for more details.
%
% You should have received a copy of the GNU General Public License along with this program. If not, see <https://www.gnu.org/licenses/>.

\documentclass[10pt,presentation]{beamer}

\usetheme[progressbar=frametitle, numbering=fraction, background=dark]{metropolis}
\usepackage{appendixnumberbeamer}

\usepackage{booktabs}
\usepackage[scale=2]{ccicons}

\usepackage{hyperref}
\usepackage{multimedia}

\usepackage{enumerate}
\usepackage[skip=2pt]{caption}

\usepackage{dirtytalk}
\usepackage{multimedia}
\usepackage[outputdir=build]{minted}

\usepackage{tikz}
\usepackage{tikzducks}
\usepackage{bookmark}

% https://tex.stackexchange.com/questions/152392/date-format-yyyy-mm-dd
\usepackage[yyyymmdd]{datetime}
\renewcommand{\dateseparator}{--}

% Copied from https://github.com/matze/mtheme/blob/master/source/beamerinnerthememetropolis.dtx
\setbeamertemplate{title page}{
  \begin{minipage}[b][\paperheight]{\textwidth}
    \vfill%
    \ifx\inserttitle\@empty\else\usebeamertemplate*{title}\fi
    \ifx\insertsubtitle\@empty\else\usebeamertemplate*{subtitle}\fi
    \usebeamertemplate*{title separator}
    \ifx\inserttitlegraphic\@empty\else\usebeamertemplate*{title graphic}\fi
    \ifx\beamer@shortauthor\@empty\else\usebeamertemplate*{author}\fi
    \ifx\insertdate\@empty\else\usebeamertemplate*{date}\fi
    \ifx\insertinstitute\@empty\else\usebeamertemplate*{institute}\fi
    \vfill
    {\tiny Copyright (C) 2023 Jamal Bouajjaj under GPLv3} \par
    \vspace*{5mm}
    \vspace*{1mm}
  \end{minipage}
}

\setbeamertemplate{title graphic}{
  \vbox to 0pt {
    \vspace*{0em}
    \inserttitlegraphic%
  }%
  \nointerlineskip%
}

\titlegraphic{\hfill \begin{tikzpicture}[scale=0.75]
\duck[magichat,magicwand,magicstars=blue!60!white,squareglasses=blue!50!black]
\end{tikzpicture}}

\newenvironment{Figure}
  {\par\medskip\noindent\minipage{\linewidth}\centering}
  {\endminipage\par\medskip}

\newcommand\Wider[2][3em]{%
\makebox[\linewidth][c]{%
  \begin{minipage}{\dimexpr\textwidth+#1\relax}
  \raggedright#2
  \end{minipage}%
  }%
}


\usepackage{media9}

\title{Lecture \#13: Threading and Queue}
\date{\today}
\author{Presented by Jamal Bouajjaj}
\institute{For University of New Haven's Fall 2023 CSCIxx51 Course}

% From https://tex.stackexchange.com/questions/66519/make-each-frame-not-slide-appear-in-the-pdf-bookmarks-with-beamer
\makeatletter
\apptocmd{\beamer@@frametitle}{\only<1>{\bookmark[page=\the\c@page,level=3]{#1}}}%
% {\message{** patching of \string\beamer@@frametitle succeeded **}}%
% {\message{** patching of \string\beamer@@frametitle failed **}}%
\makeatother

\begin{document}

\maketitle

\begin{frame}[containsverbatim]{Problem}
Let's say you have process that waits on something, for example a delay.
    \begin{minted}[breaklines=true,frame=single]{python}
import time
def do_long_thing():
    print("Doing a thing!")
    time.sleep(1)
    print("Done!")

print("Other thing")
do_long_thing()
print("Oh no, I am delayed!")
  \end{minted}
\end{frame}


\begin{frame}[containsverbatim]{Problem}
Or what if we want multiple things done together "at the same time"
    \begin{minted}[breaklines=true,frame=single]{python}
import time
def do_long_thing(i):
    print(f"Doing a thing for {i}")
    time.sleep(1)
    print("Done!")

print("Other thing")
for i in range(10):
    do_long_thing(i)
print("Oh no, I am delayed!")
  \end{minted}
\end{frame}

\section{threading}

\begin{frame}[containsverbatim]{Solved!}
Welcome to the threading module. This runs a function in a thread, allowing async functions to run while your main application is running.
    \begin{minted}[breaklines=true,frame=single]{python}
import threading
import time
def do_long_thing(i):
    print(f"Doing a thing for {i}")
    time.sleep(1)
    print("Done!")

print("Other thing")
for i in range(10):
    t = threading.Thread(target=do_long_thing, args=(i, ))
    t.start()
print("yay, not delayed!")
  \end{minted}
\end{frame}

\begin{frame}[containsverbatim]{Making one}
To make a thread, we call \verb|Thread| to make a \verb|Thread| object. Target is the function to run, and args are the arguments passed to the function given as a tuple.
  \begin{minted}[breaklines=true,frame=single]{python}
t = threading.Thread(target=TARGET, args=())
\end{minted}
This will return a Thread object
\end{frame}


\begin{frame}[containsverbatim]{Class Methods}
The following are the main methods to a thread object:
    \begin{minted}[breaklines=true,frame=single]{python}
t.start()     # starts the thread
t.is_alive()  # gets a bool depending if the thread is alive
t.join()      # waits until the thread function is exited
\end{minted}
\end{frame}

\begin{frame}[containsverbatim]{Only Once}
A Thread object can only be ran once:
    \begin{minted}[breaklines=true,frame=single]{python}
t.start()
t.join()
t.start()   # This will fail
\end{minted}
\end{frame}

\begin{frame}{Actual Threads}
When you think of a "thread", you are thinking it's a seperate process that uses another CPU core...right?\pause

What if I told you that is NOT the case!\pause

One could also say...\pause

An imposter thread \pause
\sound[channels=2,samplingrate=44100,encoding=Raw,inlinesound,autostart,automute]{AMONG US!}{audio/AMONG-US.aif}
(sorry)
\end{frame}

\section{Why?? GIL!}

\begin{frame}{GIL}
This because of the Python Global Intepreter Lock (GIL).

This internal mechanism ensures that the intepreter only executes one bytecode at a time.

This means that threading is not actually multi-CPU threaded, so your program will still run on one core.
\end{frame}

\begin{frame}{Communication}
Let's say you have a thread and GUI thread. How will you ensure nice communication between the thread and GUI?

You can just have a shared variable, but that is not thread safe, and can lead to race conditions.
\end{frame}

\section{Locks}

\begin{frame}[containsverbatim]{Lock Object}
A lock object, when called, will ensure the same lock is not executed elsewhere. It will hold the other process until the lock is released.
  \begin{minted}[breaklines=true,frame=single]{python}
tl = threading.Lock()
tl.acquire()    # Get the lock
tl.release()    # Release it back
tl.locked()     # Get if the lock is locked
with tl:  # this will acquire and release for you!
  something()
\end{minted}
\end{frame}

\section{Queue}
\begin{frame}[containsverbatim]{Lock Object}
If you want to send data back and forth, one useful thread-safe way to do so is with a Queue. This is a seperate module: queue.

A Queue is a FIFO buffer that can have stuff put into it, and stuff retreived from.
\begin{minted}[breaklines=true,frame=single]{python}
import queue
#q = queue.Queue(maxsize=0)  # maxsize is optional, can be set to limit size
q = queue.Queue()
q.put(123)
print(q.qsize())
print(q.empty())
print(q.full())   # if Queue was given a size
print(q.get())
q.join()  # Wait until all items have been grabbed.
\end{minted}
\end{frame}

\begin{frame}[standout]{End}
  The end
\end{frame}

\end{document}
