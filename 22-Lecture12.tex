% This program is free software: you can redistribute it and/or modify it under the terms of the GNU General Public License as published by the Free Software Foundation, either version 3 of the License, or (at your option) any later version.
%
% This program is distributed in the hope that it will be useful, but WITHOUT ANY WARRANTY; without even the implied warranty of MERCHANTABILITY or FITNESS FOR A PARTICULAR PURPOSE. See the GNU General Public License for more details.
%
% You should have received a copy of the GNU General Public License along with this program. If not, see <https://www.gnu.org/licenses/>.

% This program is free software: you can redistribute it and/or modify it under the terms of the GNU General Public License as published by the Free Software Foundation, either version 3 of the License, or (at your option) any later version.
%
% This program is distributed in the hope that it will be useful, but WITHOUT ANY WARRANTY; without even the implied warranty of MERCHANTABILITY or FITNESS FOR A PARTICULAR PURPOSE. See the GNU General Public License for more details.
%
% You should have received a copy of the GNU General Public License along with this program. If not, see <https://www.gnu.org/licenses/>.

\documentclass[10pt,presentation]{beamer}

\usetheme[progressbar=frametitle, numbering=fraction, background=dark]{metropolis}
\usepackage{appendixnumberbeamer}

\usepackage{booktabs}
\usepackage[scale=2]{ccicons}

\usepackage{hyperref}
\usepackage{multimedia}

\usepackage{enumerate}
\usepackage[skip=2pt]{caption}

\usepackage{dirtytalk}
\usepackage{multimedia}
\usepackage[outputdir=build]{minted}

\usepackage{tikz}
\usepackage{tikzducks}
\usepackage{bookmark}

% https://tex.stackexchange.com/questions/152392/date-format-yyyy-mm-dd
\usepackage[yyyymmdd]{datetime}
\renewcommand{\dateseparator}{--}

% Copied from https://github.com/matze/mtheme/blob/master/source/beamerinnerthememetropolis.dtx
\setbeamertemplate{title page}{
  \begin{minipage}[b][\paperheight]{\textwidth}
    \vfill%
    \ifx\inserttitle\@empty\else\usebeamertemplate*{title}\fi
    \ifx\insertsubtitle\@empty\else\usebeamertemplate*{subtitle}\fi
    \usebeamertemplate*{title separator}
    \ifx\inserttitlegraphic\@empty\else\usebeamertemplate*{title graphic}\fi
    \ifx\beamer@shortauthor\@empty\else\usebeamertemplate*{author}\fi
    \ifx\insertdate\@empty\else\usebeamertemplate*{date}\fi
    \ifx\insertinstitute\@empty\else\usebeamertemplate*{institute}\fi
    \vfill
    {\tiny Copyright (C) 2023 Jamal Bouajjaj under GPLv3} \par
    \vspace*{5mm}
    \vspace*{1mm}
  \end{minipage}
}

\setbeamertemplate{title graphic}{
  \vbox to 0pt {
    \vspace*{0em}
    \inserttitlegraphic%
  }%
  \nointerlineskip%
}

\titlegraphic{\hfill \begin{tikzpicture}[scale=0.75]
\duck[magichat,magicwand,magicstars=blue!60!white,squareglasses=blue!50!black]
\end{tikzpicture}}

\newenvironment{Figure}
  {\par\medskip\noindent\minipage{\linewidth}\centering}
  {\endminipage\par\medskip}

\newcommand\Wider[2][3em]{%
\makebox[\linewidth][c]{%
  \begin{minipage}{\dimexpr\textwidth+#1\relax}
  \raggedright#2
  \end{minipage}%
  }%
}


\title{Lecture \#12: Misc \#1}
\date{\today}
\author{Presented by Jamal Bouajjaj}
\institute{For University of New Haven's Fall 2023 CSCIxx51 Course}

% From https://tex.stackexchange.com/questions/66519/make-each-frame-not-slide-appear-in-the-pdf-bookmarks-with-beamer
\makeatletter
\apptocmd{\beamer@@frametitle}{\only<1>{\bookmark[page=\the\c@page,level=3]{#1}}}%
% {\message{** patching of \string\beamer@@frametitle succeeded **}}%
% {\message{** patching of \string\beamer@@frametitle failed **}}%
\makeatother

\begin{document}

\maketitle

\begin{frame}{Misc}
  This lecture/presentation is for a collection of stuff that I either missed, or stuff which is small to not warrant a whole class for them individually, but enough for multiple of them
\end{frame}

\section{Tuple "Collapsing"}
\begin{frame}[containsverbatim]{Tuple Collapsing}
  If you declare a tuple with one element, by default it will ignore your parenthesis, making it into a not-tuple as parenthesis can also be used to have order of operation.

  To make a one-element tuple into one, add a comma after the first element
  \begin{minted}[breaklines=true,frame=single]{python}
print(type( (1) ))   # int
print(type( (1,) ))  # tuple
  \end{minted}
\end{frame}

\section{Unpacking and Packing}
\begin{frame}[fragile]{List Unpacking}
  What if you have a Python list, where that list is actually input arguments to a function. Can you pass all of them to the function?\pause

  YES, by unpacking the list. Unpacking just means to take the elements of the sequence, and set them as arguments to a function. This is done with a star *. For example:
  \begin{minted}[breaklines=true,frame=single]{python}
def addThreeNumbers(a, b, c):
    return a + b + c
# The following are equivalent
addThreeNumbers(1, 2, 3)
addThreeNumbers(*[1, 2, 3])
  \end{minted}
\end{frame}

\begin{frame}[containsverbatim]{Dict Unpacking}
  If done with a dictionary, then the keys become the variable identifier, and the value is...well...the value of that identifier. Dictionary unpacking is done with two stars **.
  \begin{minted}[breaklines=true,frame=single]{python}
def addThreeNumbers(a, b, c):
    return a + b + c
# The following are equivalent
addThreeNumbers(a=1, b=2, c=3)
addThreeNumbers(**{'a': 1, 'b': 2, 'c': 3})
  \end{minted}
\end{frame}

\begin{frame}[fragile]{Function Packing}
  Now what if you need a function to take multiple user inputs, but have the flexiblity to have any amount. You CAN have the user just enter a list as an argument, but there is another way.\pause

  The same syntax for list unpacking sort of works backwards if it's an argument of a function: Take all of the keywords by the user, and pack them into a list. The text \textit{args} isn't fixed, but it's the standard

  \begin{minted}[breaklines=true,frame=single]{python}
def addNumbers(*args):
    return sum(args)
addNumbers(1, 2, 3)
addNumbers(1, 2, 3, 4, 5, 6, 7)
  \end{minted}
\end{frame}


\begin{frame}[containsverbatim]{Function Dict Packing}
  Same concept works for dictionaries. The standard text is \textit{kwargs}

  \begin{minted}[breaklines=true,frame=single]{python}
def addNumbers(**kwargs):
    if 'a' in kwargs:
        return 'A in args'
    return 'no a in args'
addNumbers(a=2, b=1, fes=1)
addNumbers(hb=3, bc=1, bjh=2)
  \end{minted}
\end{frame}

\begin{frame}[containsverbatim]{Why not both?}
    \begin{minted}[breaklines=true,frame=single]{python}
def addNumbers(*args, **kwargs):
    print(len(args))
    if 'a' in kwargs:
        return 'A in args'
    return 'no a in args'

addNumbers(2, 4, 1)
addNumbers(2, 4, 1, a=2, b=1, fes=1)
addNumbers(hb=3, bc=1, bjh=2)
  \end{minted}
\end{frame}

\section{Lambda}
\begin{frame}[fragile]{Lambda Function}
  A lambda function is constructor to make anonymous functions. What lambda returns is a function that can be called.

  An anonymous function (a general programming term) is a function without a name.\pause

  A $\lambda$ function has the following syntax:
  \begin{minted}[breaklines=true,frame=single]{python}
lambda *VAR: SOMETHING
  \end{minted}
  The function above takes arguments, and whatever that SOMETHING does is what the lambda function returns.
\end{frame}

\begin{frame}[containsverbatim]{Example}
\begin{minted}[breaklines=true,frame=single]{python}
def stripFunction(a):
    return a.strip()
with open('file') as f:
    r = map(stripFunction, f)
    # OR
    r = map(lambda x: x.strip(), f)
\end{minted}
\end{frame}

\begin{frame}[containsverbatim]{Other use}
  Also useful for having a function that will execute later on with arguments:
    \begin{minted}[breaklines=true,frame=single]{python}
def ret(x):
    print(x)

functions = []
for i in range(100):
    #functions.append(ret(i))   # <- not good!
    functions.append(lambda i=i: ret(i))

functions[0]()    # a bit cursed...no?
  \end{minted}
\end{frame}

\section{Block Documentation}
\begin{frame}[containsverbatim]{Block}
If you need multiline documentation, you need need individual \#. OR, you can just surround your comment in 3x"
\begin{minted}[breaklines=true,frame=single]{python}
"""
This is a block documentation

Anything in here is a comment
"""

"""Another valid block doc: don't have to be multi-line"""
\end{minted}
\end{frame}

\begin{frame}[containsverbatim]{Reservations}
With that said, the convention is to keep block documentation only for function, class, or module docs. These are called \textit{docstrings}
\begin{minted}[breaklines=true,frame=single]{python}
def your_function(a, b):
  """
  This functoin returns if a > b

  Args:
    a: number
    b: number
  """
  return a > b
\end{minted}
\end{frame}

\begin{frame}{Formats}
There are multiple docstring format conventions if you want to follow them. They tend to include all of a function's info like arguments, return type, exceptions, etc. They can also be used to automatically generate a documentation webpage.

See \url{https://stackoverflow.com/questions/3898572/what-are-the-most-common-python-docstring-formats} for the different formats.

PyCharm can handle all of them, and will auto-fill a docstring for a function when you make one.
\end{frame}

\begin{frame}[containsverbatim]{Getting docstrings}
You can also have Python return a docstring of a function or class by calling \verb|__doc__|:
\begin{minted}[breaklines=true,frame=single]{python}
import random
print(random.randint.__doc__)
\end{minted}
\end{frame}

\section{Type Hinting}
\begin{frame}[containsverbatim]{Hinting}
If you have a function as follows, let's say it's expecting a certain input type
\begin{minted}[breaklines=true,frame=single]{python}
def really_cool_function(money):
  if money < 50:
    print("You broke")
  else:
    print("You not broke")
\end{minted}
And this will fail if the type isn't a float or integer. How do you convey it?
\end{frame}

\begin{frame}[containsverbatim]{Docs}
You could have the block documentation state so
\begin{minted}[breaklines=true,frame=single]{python}
def really_cool_function(money):
  """
  Really cool function

  Args:
    money (int): This is an integer!
  """
  if money < 50:
    print("You broke")
  else:
    print("You not broke")
\end{minted}
But what if there was a BETTER way, one in which the IDE can also understand and do static checking to that who uses this function?
\end{frame}

\begin{frame}[containsverbatim]{Hinting}
Introducing type hinting! Now with this one small trick (colon), you can have the user, IDE, and any static checker know what type your function is expecting!

After a variable, you type colon with the type class it expects
\begin{minted}[breaklines=true,frame=single]{python}
def really_cool_function(money: int):
  if money < 50:
    print("You broke")
  else:
    print("You not broke")
\end{minted}
\end{frame}

\begin{frame}[containsverbatim]{Hinting 2}
Also works with optional variables, before the equate sign.
\begin{minted}[breaklines=true,frame=single]{python}
def really_cool_function2(money: int, areyoucool: bool = False):
  if areyoucool:
    print("You're cool anyways, who needs money!")
    return
  if money < 50:
    print("You broke")
  else:
    print("You not broke")
\end{minted}
\end{frame}

\section{Global}

\begin{frame}[containsverbatim]{Variable Scope}
  Variables have a "scope" to them, i.e what part of code they encompass.

  For example, a variable declared in a function has it's scope withing the function, and cannot be accessed externally:
\begin{minted}[breaklines=true,frame=single]{python}
def really_cool_function3():
    money = 5

really_cool_function3()
print(money)    # This will fail
\end{minted}
\end{frame}

\begin{frame}[containsverbatim]{Variable Scope}
  But a function can access outside scope variables if it is not defined
\begin{minted}[breaklines=true,frame=single]{python}
def really_cool_function4():
    print(cash)

cash = 50
really_cool_function4()
\end{minted}
\end{frame}

\begin{frame}[containsverbatim]{Variable Scope}
  If function uses a local variable, even if later, then the global variable is not used:
\begin{minted}[breaklines=true,frame=single]{python}
def really_cool_function4():
    print(cash) # this will fail
    cash = 10

cash = 50
really_cool_function4()
\end{minted}
\end{frame}

\begin{frame}[containsverbatim]{Global}
  Unless you have a \verb|global| keyword to have the scope of the variable be outside of the function. This will also allow functions to change global variables
\begin{minted}[breaklines=true,frame=single]{python}
def really_cool_function4():
    global cash
    print(cash)
    cash = 10
    print(cash)

cash = 50
print(cash)
really_cool_function4()
print(cash)
\end{minted}
\end{frame}

\begin{frame}[containsverbatim]{Guidance}
  My advice: \textit{AVOID globals if possible}.

  They can lead to unexpected bugs, and aren't a good design practice. For larger/robust applications, each function should have a defined input and output. If some variables needs to be kept in a state, use classes.
\end{frame}

\begin{frame}[standout]{End}
  The end
\end{frame}

\end{document}
