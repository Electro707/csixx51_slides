% This program is free software: you can redistribute it and/or modify it under the terms of the GNU General Public License as published by the Free Software Foundation, either version 3 of the License, or (at your option) any later version.
%
% This program is distributed in the hope that it will be useful, but WITHOUT ANY WARRANTY; without even the implied warranty of MERCHANTABILITY or FITNESS FOR A PARTICULAR PURPOSE. See the GNU General Public License for more details.
%
% You should have received a copy of the GNU General Public License along with this program. If not, see <https://www.gnu.org/licenses/>.

% This program is free software: you can redistribute it and/or modify it under the terms of the GNU General Public License as published by the Free Software Foundation, either version 3 of the License, or (at your option) any later version.
%
% This program is distributed in the hope that it will be useful, but WITHOUT ANY WARRANTY; without even the implied warranty of MERCHANTABILITY or FITNESS FOR A PARTICULAR PURPOSE. See the GNU General Public License for more details.
%
% You should have received a copy of the GNU General Public License along with this program. If not, see <https://www.gnu.org/licenses/>.

\documentclass[10pt,presentation]{beamer}

\usetheme[progressbar=frametitle, numbering=fraction, background=dark]{metropolis}
\usepackage{appendixnumberbeamer}

\usepackage{booktabs}
\usepackage[scale=2]{ccicons}

\usepackage{hyperref}
\usepackage{multimedia}

\usepackage{enumerate}
\usepackage[skip=2pt]{caption}

\usepackage{dirtytalk}
\usepackage{multimedia}
\usepackage[outputdir=build]{minted}

\usepackage{tikz}
\usepackage{tikzducks}
\usepackage{bookmark}

% https://tex.stackexchange.com/questions/152392/date-format-yyyy-mm-dd
\usepackage[yyyymmdd]{datetime}
\renewcommand{\dateseparator}{--}

% Copied from https://github.com/matze/mtheme/blob/master/source/beamerinnerthememetropolis.dtx
\setbeamertemplate{title page}{
  \begin{minipage}[b][\paperheight]{\textwidth}
    \vfill%
    \ifx\inserttitle\@empty\else\usebeamertemplate*{title}\fi
    \ifx\insertsubtitle\@empty\else\usebeamertemplate*{subtitle}\fi
    \usebeamertemplate*{title separator}
    \ifx\inserttitlegraphic\@empty\else\usebeamertemplate*{title graphic}\fi
    \ifx\beamer@shortauthor\@empty\else\usebeamertemplate*{author}\fi
    \ifx\insertdate\@empty\else\usebeamertemplate*{date}\fi
    \ifx\insertinstitute\@empty\else\usebeamertemplate*{institute}\fi
    \vfill
    {\tiny Copyright (C) 2023 Jamal Bouajjaj under GPLv3} \par
    \vspace*{5mm}
    \vspace*{1mm}
  \end{minipage}
}

\setbeamertemplate{title graphic}{
  \vbox to 0pt {
    \vspace*{0em}
    \inserttitlegraphic%
  }%
  \nointerlineskip%
}

\titlegraphic{\hfill \begin{tikzpicture}[scale=0.75]
\duck[magichat,magicwand,magicstars=blue!60!white,squareglasses=blue!50!black]
\end{tikzpicture}}

\newenvironment{Figure}
  {\par\medskip\noindent\minipage{\linewidth}\centering}
  {\endminipage\par\medskip}

\newcommand\Wider[2][3em]{%
\makebox[\linewidth][c]{%
  \begin{minipage}{\dimexpr\textwidth+#1\relax}
  \raggedright#2
  \end{minipage}%
  }%
}


\title{Lecture \#2: Functions, Mutability, Input, and Strings}
\date{\today}
\author{Presented by Jamal Bouajjaj}
\institute{For University of New Haven's Fall 2023 CSCIxx51 Course}

\begin{document}

\maketitle

\section{Functions}
\begin{frame}[containsverbatim]{Functions}
    Python functions are just like any other programming language's functions. The following is the general syntax:

    \begin{minted}[breaklines=true,frame=single]{text}
def iamafunction(ARG1, ARG2, ARG3=None, ARG4=b):
    something
    return OUTPUT
    \end{minted}
\end{frame}

\begin{frame}{Arugments}
    Arguments are optional. There can be multiple, and there are optional arguments that default to something if not passed to the function.
\end{frame}

\begin{frame}{Returns}
    Returns are also optional. There can also be multiple things returned, returning as a tuple
\end{frame}

\section{Mutable vs Immutable}
\begin{frame}[containsverbatim]{What?}
    A Immutable variable is one that can cannot change it's value after creation, while a multable variable is one that can
\end{frame}

\begin{frame}[containsverbatim]{Examples}
    I will show some examples. To know if one is multable or not by re-assignment, one can check it's ID
\end{frame}

\section{Sequences}

\begin{frame}[containsverbatim]{List}
    A sequences is a positionally ordered collection of items. For example, a list:
    \begin{minted}[breaklines=true,frame=single]{python}
a = [2, 5, 7, 3, 6, 8, 6]
    \end{minted}
\end{frame}

\begin{frame}[containsverbatim]{Tuple}
    A tuples is a immutable listm, i.e cannot be changed after it's creation
    \begin{minted}[breaklines=true,frame=single]{python}
a = (2, 5, 7, 3, 6, 8, 6)
# This should throw an error
a[2] = 34
    \end{minted}
\end{frame}

\begin{frame}[containsverbatim]{Sets}
    A set an un-orderd list that is immutable

    It is the least used type
    \begin{minted}[breaklines=true,frame=single]{python}
a = {2, 5, 7, 3, 6, 8, 6}
b = {5, 2}
# This should throw an error
print(a[5])
print(a-b)
    \end{minted}
\end{frame}

\begin{frame}[containsverbatim]{Strings}
    A string is also a sequence by itself
    \begin{minted}[breaklines=true,frame=single]{python}
a = "Hello World!"
print(a[2])
    \end{minted}
\end{frame}

\begin{frame}[containsverbatim]{Bytearray}
    A binary list
    \begin{minted}[breaklines=true,frame=single]{python}
a = bytearray([234, 12, 34, 65])
    \end{minted}
\end{frame}

\begin{frame}[containsverbatim]{Sequence Operations}
    There are a couple of methods one can do with sequences, such as (shown on screen)
\end{frame}

\begin{frame}[containsverbatim]{Sequence Indexing}
    One can index a Python list with \mintinline{python}{[START:END]}. You can also have negatives
\end{frame}

\section{Dictionaries}

\begin{frame}[containsverbatim]{What?}
    A variable to map keys to values. Think of it as a list but you choose the index
    \begin{minted}[breaklines=true,frame=single]{python}
a = {'resistor': 1E3, 'tollerance': '+-yes'}
    \end{minted}
\end{frame}

\begin{frame}[containsverbatim]{No Duplicates}
    A dictionary cannot have the same key
    \begin{minted}[breaklines=true,frame=single]{python}
# This cause the first `number` key set to be ignored
a = {'number': 1E3, 'number': 34, 'fruit': 'banana'}
    \end{minted}
\end{frame}

\section{Input}

\begin{frame}[containsverbatim]{Input}
    The input function allows you to type text into your application
    \begin{minted}[breaklines=true,frame=single]{python}
a = input("Please enter something :) ")
    \end{minted}
\end{frame}

\section{String Formatting}
\begin{frame}[containsverbatim]{Variables in Strings?}
    Python strings can be formatted to add text, similar to printf
    \begin{minted}[breaklines=true,frame=single]{python}
a = 123
# The following are all the same, from least to most preferable
print("This is a number %s" % a)
print("This is a number {:d}".format(a))
print(f"This is a number {a:d}")       # Since Python 3.6
    \end{minted}
\end{frame}

\section{Formatting List}
\begin{frame}[containsverbatim]{Available Formatting}
    The format can be found in the docs, but here are a couple
    \begin{minted}[breaklines=true,frame=single]{python}
print(f"Numbers: {123:d}")
print(f"Numbers: {123:05d}")
print(f"Numbers: {123:x}")
print(f"Numbers: {123.0:.2f}")
print(f"Numbers: {0.0000000000562:g}")
print(f"Numbers: {"Text!":s}")
    \end{minted}
\end{frame}

\begin{frame}[standout]{End}
  The end
\end{frame}

\end{document}
