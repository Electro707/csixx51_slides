% This program is free software: you can redistribute it and/or modify it under the terms of the GNU General Public License as published by the Free Software Foundation, either version 3 of the License, or (at your option) any later version.
%
% This program is distributed in the hope that it will be useful, but WITHOUT ANY WARRANTY; without even the implied warranty of MERCHANTABILITY or FITNESS FOR A PARTICULAR PURPOSE. See the GNU General Public License for more details.
%
% You should have received a copy of the GNU General Public License along with this program. If not, see <https://www.gnu.org/licenses/>.

% This program is free software: you can redistribute it and/or modify it under the terms of the GNU General Public License as published by the Free Software Foundation, either version 3 of the License, or (at your option) any later version.
%
% This program is distributed in the hope that it will be useful, but WITHOUT ANY WARRANTY; without even the implied warranty of MERCHANTABILITY or FITNESS FOR A PARTICULAR PURPOSE. See the GNU General Public License for more details.
%
% You should have received a copy of the GNU General Public License along with this program. If not, see <https://www.gnu.org/licenses/>.

\documentclass[10pt,presentation]{beamer}

\usetheme[progressbar=frametitle, numbering=fraction, background=dark]{metropolis}
\usepackage{appendixnumberbeamer}

\usepackage{booktabs}
\usepackage[scale=2]{ccicons}

\usepackage{hyperref}
\usepackage{multimedia}

\usepackage{enumerate}
\usepackage[skip=2pt]{caption}

\usepackage{dirtytalk}
\usepackage{multimedia}
\usepackage[outputdir=build]{minted}

\usepackage{tikz}
\usepackage{tikzducks}
\usepackage{bookmark}

% https://tex.stackexchange.com/questions/152392/date-format-yyyy-mm-dd
\usepackage[yyyymmdd]{datetime}
\renewcommand{\dateseparator}{--}

% Copied from https://github.com/matze/mtheme/blob/master/source/beamerinnerthememetropolis.dtx
\setbeamertemplate{title page}{
  \begin{minipage}[b][\paperheight]{\textwidth}
    \vfill%
    \ifx\inserttitle\@empty\else\usebeamertemplate*{title}\fi
    \ifx\insertsubtitle\@empty\else\usebeamertemplate*{subtitle}\fi
    \usebeamertemplate*{title separator}
    \ifx\inserttitlegraphic\@empty\else\usebeamertemplate*{title graphic}\fi
    \ifx\beamer@shortauthor\@empty\else\usebeamertemplate*{author}\fi
    \ifx\insertdate\@empty\else\usebeamertemplate*{date}\fi
    \ifx\insertinstitute\@empty\else\usebeamertemplate*{institute}\fi
    \vfill
    {\tiny Copyright (C) 2023 Jamal Bouajjaj under GPLv3} \par
    \vspace*{5mm}
    \vspace*{1mm}
  \end{minipage}
}

\setbeamertemplate{title graphic}{
  \vbox to 0pt {
    \vspace*{0em}
    \inserttitlegraphic%
  }%
  \nointerlineskip%
}

\titlegraphic{\hfill \begin{tikzpicture}[scale=0.75]
\duck[magichat,magicwand,magicstars=blue!60!white,squareglasses=blue!50!black]
\end{tikzpicture}}

\newenvironment{Figure}
  {\par\medskip\noindent\minipage{\linewidth}\centering}
  {\endminipage\par\medskip}

\newcommand\Wider[2][3em]{%
\makebox[\linewidth][c]{%
  \begin{minipage}{\dimexpr\textwidth+#1\relax}
  \raggedright#2
  \end{minipage}%
  }%
}


\usepackage{media9}
\usepackage{multicol}

\title{Lecture \#14: GUI: Tkinter}
\date{\today}
\author{Presented by Jamal Bouajjaj}
\institute{For University of New Haven's Fall 2023 CSCIxx51 Course}

% From https://tex.stackexchange.com/questions/66519/make-each-frame-not-slide-appear-in-the-pdf-bookmarks-with-beamer
\makeatletter
\apptocmd{\beamer@@frametitle}{\only<1>{\bookmark[page=\the\c@page,level=3]{#1}}}%
% {\message{** patching of \string\beamer@@frametitle succeeded **}}%
% {\message{** patching of \string\beamer@@frametitle failed **}}%
\makeatother

\begin{document}

\maketitle

\begin{frame}{Interface}
There are 3 "interface types" that are possible with a modern computer
\begin{itemize}
  \item CLI: Command-Line Interface. Interface with the application thru a command line sending text to it
  \item TUI: Terminal User Interface. Interface with the application thru a terminal, but interactivally
  \item GUI: Graphical User Interface. Interface with the application with graphics
\end{itemize}
So far, we have only done CLI. We will be going over GUI, but not TUI (if you're intrested, look into \textit{curses}
\end{frame}


\begin{frame}{Widget Toolkit}
A \textit{Widget Toolkit} or \textit{GUI Framework} is a library (or multiple) that allows you to make a GUI application. A lot of them are OS-independent.

There are many toolkits for Python, including:
\begin{itemize}
  \item Tkinter (what we will be using)
  \item PyQT/PySide
  \item wxPython
  \item Kivy
\end{itemize}

\footnote{\url{https://en.wikipedia.org/wiki/Widget_toolkit}}
\end{frame}

\section{tkinter}

\begin{frame}{tkinter}
\textit{tkinter} is the defacto-standard toolkit for making simple GUI in Python. So much in fact that Python ships with it as a standard library.

\textit{tkinter} is technically a Python wrapper for \textit{tk}, a GUI framework that is ran under the \textit{tcl} programming language.

There is also \textit{tkinter.ttk}, which just has themed widgets, and is also part of the \textit{tk} library.
\end{frame}

\begin{frame}{Widgets}
A widget is some GUI element that can be used as an element of interaction. Think like buttons, scroll-bars, labels, etc.
Tk has many widgets, all in the official docs. Here are some common/popular ones
\begin{multicols}{2}
\begin{itemize}
  \item label
  \item button
  \item listbox
  \item checkbutton
  \item text
  \item radiobox
  \item canvas
\end{itemize}
AND MORE!
\end{multicols}
\end{frame}

\begin{frame}[containsverbatim]{Invoking}
To get started, we need to make a top-level (a window) that is the root, the top-most level. Then we can run it
    \begin{minted}[breaklines=true,frame=single]{python}
import tkinter as tk
# The top-level window
root = tk.Tk()
# This runs TK. This will HANG FOREVER until the GUI is closed
root.mainloop()
  \end{minted}
\end{frame}

\begin{frame}[containsverbatim]{Widget}
Let's add a widget
    \begin{minted}[breaklines=true,frame=single,fontsize=\small]{python}
import tkinter as tk
# The top-level window
root = tk.Tk()

label = tk.Label(root, text="Welcome World!")
label.pack()   # We place the label (see next slide)

# This runs TK. This will HANG FOREVER until the GUI is closed
root.mainloop()
  \end{minted}
\end{frame}

\begin{frame}[containsverbatim]{Placement}
How do we place widgets? Tkinter offers 3 "styles.
\begin{itemize}
  \item pack(): Places widgets on top of one another, or side to side.
  \item grid(): Places the widget in a grid-like fashion
  \item place(): Places the widget in direct x-y coordinates in the window
\end{itemize}
You must stick to one placement style per frame/top-level.
\end{frame}

\begin{frame}[containsverbatim]{Binding}
Each widget can have events binded to it, such as a key press, button press, or other events. Each binding call need to take one argument, which is the event
    \begin{minted}[breaklines=true,frame=single,fontsize=\small]{python}
    root = tk.Tk()

    text = tk.Label(root, text="A thing!")
    text.pack()

    text.bind("<Enter>", lambda x: print("Entered"))
    text.bind("<Leave>", lambda x: print("Left"))

    root.bind("q", lambda x: root.destroy())

    # run
    root.mainloop()
  \end{minted}
\end{frame}

\begin{frame}[containsverbatim]{Messagebox}
There are several built-in message box, such as
  \begin{minted}[breaklines=true,frame=single,fontsize=\small]{python}
from tkinter import messagebox
# The top-level window
messagebox.showinfo("TITLE", "MESSAGE")
messagebox.showwarning("TITLE", "MESSAGE")
messagebox.showerror("TITLE", "MESSAGE")
messagebox.askquestion("TITLE", "MESSAGE")
messagebox.askokcancel("TITLE", "MESSAGE")
messagebox.askretrycancel("TITLE", "MESSAGE")
messagebox.askyesno("TITLE", "MESSAGE")
messagebox.askyesnocancel("TITLE", "MESSAGE")
  \end{minted}
\end{frame}

\begin{frame}[containsverbatim]{Dialogs}
There are several built-in dialogs into tkinter, such as
  \begin{minted}[breaklines=true,frame=single,fontsize=\small]{python}
from tkinter import simpledialog
# The top-level window
simpledialog.askfloat("TITLE", "MESSAGE")
simpledialog.askinteger("TITLE", "MESSAGE")
simpledialog.askstring("TITLE", "MESSAGE")
  \end{minted}
  Also file dialogs
  \begin{minted}[breaklines=true,frame=single,fontsize=\small]{python}
from tkinter import filedialog
# Actually return a FileIO
filedialog.askopenfile()
filedialog.asksaveasfile()
# Only return file name
filedialog.askopenfilename()
filedialog.asksaveasfilename()
# Ask for directory
filedialog.askdirectory()
  \end{minted}
\end{frame}

\begin{frame}{References}
\begin{itemize}
  \item \url{https://www.tcl.tk/man/tcl8.6/TkCmd/contents.html}
  \item \url{https://docs.python.org/3/library/tk.html}
  \item \url{https://anzeljg.github.io/rin2/book2/2405/docs/tkinter/index.html}
\end{itemize}
\end{frame}

\begin{frame}[standout]{End}
  The end. \pause

  Wait, don't we need to know more?\pause

  Well yes, but I will be doing that as an in-class demo/live coding
\end{frame}

\end{document}
