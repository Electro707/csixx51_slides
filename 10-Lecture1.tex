% This program is free software: you can redistribute it and/or modify it under the terms of the GNU General Public License as published by the Free Software Foundation, either version 3 of the License, or (at your option) any later version.
%
% This program is distributed in the hope that it will be useful, but WITHOUT ANY WARRANTY; without even the implied warranty of MERCHANTABILITY or FITNESS FOR A PARTICULAR PURPOSE. See the GNU General Public License for more details.
%
% You should have received a copy of the GNU General Public License along with this program. If not, see <https://www.gnu.org/licenses/>.

\documentclass[10pt]{beamer}

\usetheme[progressbar=frametitle, numbering=fraction, background=dark]{metropolis}
\usepackage{appendixnumberbeamer}

\usepackage{booktabs}
\usepackage[scale=2]{ccicons}

\usepackage{hyperref}
\usepackage{multimedia}

\usepackage{enumerate}
\usepackage[skip=2pt]{caption}

\usepackage{multimedia}
\usepackage{minted}

% This program is free software: you can redistribute it and/or modify it under the terms of the GNU General Public License as published by the Free Software Foundation, either version 3 of the License, or (at your option) any later version.
%
% This program is distributed in the hope that it will be useful, but WITHOUT ANY WARRANTY; without even the implied warranty of MERCHANTABILITY or FITNESS FOR A PARTICULAR PURPOSE. See the GNU General Public License for more details.
%
% You should have received a copy of the GNU General Public License along with this program. If not, see <https://www.gnu.org/licenses/>.

\documentclass[10pt,presentation]{beamer}

\usetheme[progressbar=frametitle, numbering=fraction, background=dark]{metropolis}
\usepackage{appendixnumberbeamer}

\usepackage{booktabs}
\usepackage[scale=2]{ccicons}

\usepackage{hyperref}
\usepackage{multimedia}

\usepackage{enumerate}
\usepackage[skip=2pt]{caption}

\usepackage{dirtytalk}
\usepackage{multimedia}
\usepackage[outputdir=build]{minted}

\usepackage{tikz}
\usepackage{tikzducks}
\usepackage{bookmark}

% https://tex.stackexchange.com/questions/152392/date-format-yyyy-mm-dd
\usepackage[yyyymmdd]{datetime}
\renewcommand{\dateseparator}{--}

% Copied from https://github.com/matze/mtheme/blob/master/source/beamerinnerthememetropolis.dtx
\setbeamertemplate{title page}{
  \begin{minipage}[b][\paperheight]{\textwidth}
    \vfill%
    \ifx\inserttitle\@empty\else\usebeamertemplate*{title}\fi
    \ifx\insertsubtitle\@empty\else\usebeamertemplate*{subtitle}\fi
    \usebeamertemplate*{title separator}
    \ifx\inserttitlegraphic\@empty\else\usebeamertemplate*{title graphic}\fi
    \ifx\beamer@shortauthor\@empty\else\usebeamertemplate*{author}\fi
    \ifx\insertdate\@empty\else\usebeamertemplate*{date}\fi
    \ifx\insertinstitute\@empty\else\usebeamertemplate*{institute}\fi
    \vfill
    {\tiny Copyright (C) 2023 Jamal Bouajjaj under GPLv3} \par
    \vspace*{5mm}
    \vspace*{1mm}
  \end{minipage}
}

\setbeamertemplate{title graphic}{
  \vbox to 0pt {
    \vspace*{0em}
    \inserttitlegraphic%
  }%
  \nointerlineskip%
}

\titlegraphic{\hfill \begin{tikzpicture}[scale=0.75]
\duck[magichat,magicwand,magicstars=blue!60!white,squareglasses=blue!50!black]
\end{tikzpicture}}

\newenvironment{Figure}
  {\par\medskip\noindent\minipage{\linewidth}\centering}
  {\endminipage\par\medskip}

\newcommand\Wider[2][3em]{%
\makebox[\linewidth][c]{%
  \begin{minipage}{\dimexpr\textwidth+#1\relax}
  \raggedright#2
  \end{minipage}%
  }%
}


\title{Lecture \#1: Where it starts!}
\author{By Jamal Bouajjaj}
\date{\today}
\institute{For University of New Haven's Fall 2023 CSCIxx51 Course}

\begin{document}

\maketitle

\begin{frame}[standout]{First Slide}
  Welcome to Script Programming/Python
\end{frame}

\section{Syllabus}

\begin{frame}{The Syllabus}
    Placeholder
\end{frame}

\section{Installation}

\section{Intro to Python}
\begin{frame}{What is it?}
    Python is a dynamially-typed, high-level, intepreted, script-able language
\end{frame}

\begin{frame}{High-Level}
    Python does a lot of heavy lifting for you!

    Memory managment for example is not heard of in Python-land, it allocates the memory it needs for you and collects your garbage.

    Generally, it's easier to do tasks in Python in C/C++ for example.

    All this goodness has a sacrifice: speed.
\end{frame}

\begin{frame}{Intepreted?}
    Running a Python line does not require what is before it to be compiled down.

    Sort of just runs the application. Exectues your application line by line as it comes.
\end{frame}

\begin{frame}[containsverbatim]{Intepreted Caveat}
    Because Python is intepreted and dynamially typed, it will not check some errors ahead of time like undeclared variables

    \begin{minted}[breaklines=true,frame=single]{python}
x = input("Please don't type 0 ")
if x == "0":
    fprint("How could you")
else:
    print("Yeah, we're happy!")
    \end{minted}

    Python must understand what it's given, but it makes no assumption of previous declarations.
\end{frame}

\begin{frame}{Implementation}
    Just like C, Python is a language specification. Like the different compilers, there are different Python intepreters.

    The most popular and widely used, CPython, is the one that you just downloaded. It's what people just refer to as "Python". It's also the reference implementation.

    There are several other intepreters, some of which are:
    \begin{itemize}
        \item Jython: A Java implementation with nice integration with Java bytecode
        \item PyPy: Implementation that uses a Just-In-Time compiler similar to Java, is faster
        \item RustPython: A Rust implementation
    \end{itemize}
\end{frame}

\begin{frame}{Syntax}
    We'll learn as we go along for the most part, but there are a couple of things I would like to note:
    \begin{itemize}
        \item It enforces whitespace indentation, so indenting is required. No more cursed one-lined obfuscated JS
        \item The whitespace can be space or tabs, but must be consistent
        \item The recommended is 4 spaces
        \item While the intepreter doesn't force you, there is a consensus on naming convensions (PEP 8). I personally will follow PEP 8
    \end{itemize}
\end{frame}

\begin{frame}{Variable Types}
    There are some variable types, some of which we'll cover later in the course. Here are some basic ones:
    \begin{itemize}
        \item integers
        \item boolean: True or False
        \item floating point numbers
        \item complex numbers: yes Python has native support for those
        \item strings
        \item lists
        \item etc (not a type)
    \end{itemize}
\end{frame}

\begin{frame}[standout]{End}
  The end
\end{frame}

\end{document}
