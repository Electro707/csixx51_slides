% This program is free software: you can redistribute it and/or modify it under the terms of the GNU General Public License as published by the Free Software Foundation, either version 3 of the License, or (at your option) any later version.
%
% This program is distributed in the hope that it will be useful, but WITHOUT ANY WARRANTY; without even the implied warranty of MERCHANTABILITY or FITNESS FOR A PARTICULAR PURPOSE. See the GNU General Public License for more details.
%
% You should have received a copy of the GNU General Public License along with this program. If not, see <https://www.gnu.org/licenses/>.

% This program is free software: you can redistribute it and/or modify it under the terms of the GNU General Public License as published by the Free Software Foundation, either version 3 of the License, or (at your option) any later version.
%
% This program is distributed in the hope that it will be useful, but WITHOUT ANY WARRANTY; without even the implied warranty of MERCHANTABILITY or FITNESS FOR A PARTICULAR PURPOSE. See the GNU General Public License for more details.
%
% You should have received a copy of the GNU General Public License along with this program. If not, see <https://www.gnu.org/licenses/>.

\documentclass[10pt,presentation]{beamer}

\usetheme[progressbar=frametitle, numbering=fraction, background=dark]{metropolis}
\usepackage{appendixnumberbeamer}

\usepackage{booktabs}
\usepackage[scale=2]{ccicons}

\usepackage{hyperref}
\usepackage{multimedia}

\usepackage{enumerate}
\usepackage[skip=2pt]{caption}

\usepackage{dirtytalk}
\usepackage{multimedia}
\usepackage[outputdir=build]{minted}

\usepackage{tikz}
\usepackage{tikzducks}
\usepackage{bookmark}

% https://tex.stackexchange.com/questions/152392/date-format-yyyy-mm-dd
\usepackage[yyyymmdd]{datetime}
\renewcommand{\dateseparator}{--}

% Copied from https://github.com/matze/mtheme/blob/master/source/beamerinnerthememetropolis.dtx
\setbeamertemplate{title page}{
  \begin{minipage}[b][\paperheight]{\textwidth}
    \vfill%
    \ifx\inserttitle\@empty\else\usebeamertemplate*{title}\fi
    \ifx\insertsubtitle\@empty\else\usebeamertemplate*{subtitle}\fi
    \usebeamertemplate*{title separator}
    \ifx\inserttitlegraphic\@empty\else\usebeamertemplate*{title graphic}\fi
    \ifx\beamer@shortauthor\@empty\else\usebeamertemplate*{author}\fi
    \ifx\insertdate\@empty\else\usebeamertemplate*{date}\fi
    \ifx\insertinstitute\@empty\else\usebeamertemplate*{institute}\fi
    \vfill
    {\tiny Copyright (C) 2023 Jamal Bouajjaj under GPLv3} \par
    \vspace*{5mm}
    \vspace*{1mm}
  \end{minipage}
}

\setbeamertemplate{title graphic}{
  \vbox to 0pt {
    \vspace*{0em}
    \inserttitlegraphic%
  }%
  \nointerlineskip%
}

\titlegraphic{\hfill \begin{tikzpicture}[scale=0.75]
\duck[magichat,magicwand,magicstars=blue!60!white,squareglasses=blue!50!black]
\end{tikzpicture}}

\newenvironment{Figure}
  {\par\medskip\noindent\minipage{\linewidth}\centering}
  {\endminipage\par\medskip}

\newcommand\Wider[2][3em]{%
\makebox[\linewidth][c]{%
  \begin{minipage}{\dimexpr\textwidth+#1\relax}
  \raggedright#2
  \end{minipage}%
  }%
}


\title{Lecture \#6: Everything's an Object!}
\date{\today}
\author{Presented by Jamal Bouajjaj}
\institute{For University of New Haven's Fall 2023 CSCIxx51 Course}

% From https://tex.stackexchange.com/questions/66519/make-each-frame-not-slide-appear-in-the-pdf-bookmarks-with-beamer
\makeatletter
\apptocmd{\beamer@@frametitle}{\only<1>{\bookmark[page=\the\c@page,level=3]{#1}}}%
% {\message{** patching of \string\beamer@@frametitle succeeded **}}%
% {\message{** patching of \string\beamer@@frametitle failed **}}%
\makeatother

\begin{document}

\maketitle

\begin{frame}[containsverbatim]{Definitions}
  A \textit{namespace} is the mapping from a name to an object, for example \mintinline{python}{a = 5}, \verb|a| is the namespace for the object  that was just created.
\end{frame}

\section{Objects?}

\begin{frame}{Objects!}
  On a core programming level, an object is a \textit{thing} you can intract with. Similar to C++
\end{frame}

\begin{frame}{POP QUIZ}
  POP QUIZ: Are the following an object?
  \begin{itemize}
    \item Class Instances: \pause YES \pause
    \item Functions: \pause YES \pause
    \item Lists: \pause YES \pause
    \item Strings: \pause YES (trick question, same as list: sequences) \pause
    \item Integers?: \pause YES \pause
    \item You???: \pause wat
  \end{itemize}
\end{frame}

\begin{frame}{You PASS}
  That's right, EVERYTHING in Python is an object and is treated as such.
\end{frame}

\begin{frame}[containsverbatim]{Object Attributes}
  Each object has attributes (or attribube references), which can be variables, function, or more! The attributes can be called with the dot syntax:

  \mintinline{python}{object.attriube}
\end{frame}

\begin{frame}[containsverbatim]{Wait...we've seen this before}
  Remember things \mintinline{python}{"Hello!".endswith('1')}? Well that is because \mintinline{python}{"Hello!"} is a string object, and \mintinline{python}{endswith()} is an attribube of that function.
\end{frame}

\section{Classes}

\begin{frame}[containsverbatim]{Definition}
  Classes is a building block to making your own objects.

  To make your own class, we can define one like below as an example
\begin{minted}[breaklines=true,frame=single]{python}
class A:
  b = 5

  def test(self):
    return "test good"
\end{minted}

Where \verb|b| and \verb|test| are \textit{attributes} of that class.
\end{frame}

\begin{frame}[containsverbatim]{Class Instance}
  After declaration of your class, you can create instances of the class

\begin{minted}[breaklines=true,frame=single]{python}
b = A() # instance
print(b.b)
b.test()
\end{minted}
\end{frame}

\begin{frame}[containsverbatim]{Variables}
  A class variables is one that is defined in the class, and is shared by all classes
\begin{minted}[breaklines=true,frame=single]{python}
class Fruit:
  is_fruit = True

orange = Fruit()
apple = Fruit()

print(orange.is_fruit)
print(apple.is_fruit)
\end{minted}
\end{frame}

\begin{frame}[containsverbatim]{Variables Caveat}
  There is a caveat when objects are not re-definable (i.e mutable)
\begin{minted}[breaklines=true,frame=single]{python}
class Fruit:
  fruit_types = []
  name = None

orange = Fruit()
apple = Fruit()

orange.name = "orange"    # New object is created
orange.fruit_types.append('orange') # wait a minute...
print(apple.name)                   # ...
print(apple.fruit_types)            # wat
\end{minted}
We'll get to how to fix this soon!
\end{frame}

\begin{frame}[containsverbatim]{Instance Objects}
  In classes, you can have instance-specific objects refered to with \verb|self| in the class.
\begin{minted}[fontsize=\small,breaklines=true,frame=single]{python}
class Fruit:
  def append_fruit(self, f_name):
    if not hasattr(self, 'b'):
      self.b = []
    self.b.append(f_name)

orange = Fruit()
orange.append_fruit('orange')
apple = Fruit()
apple.append_fruit('apple')
orange.append_fruit('orange2')

print(apple.b)
\end{minted}
\end{frame}

\begin{frame}[containsverbatim]{Functions and Self}
  Each class function is automatically given one argument: \verb|self|. This refers to the object's own instance.

  You don't pass in a \verb|self| when you call the class instance's function

  Technically it's just an argument, so it doesn't have to be named \verb|self|, but that is convention.
\end{frame}


\begin{frame}[containsverbatim]{Special Class Name}
  In Python, there are some class names that gets called automatically depending on what is done to the object or around it, which are \textit{special names}.

  The most popular one is \verb|__init__()|, where it gets called when a class object is initialized, and is useful to define some stuff
\end{frame}

\begin{frame}[containsverbatim]{Init}
  \begin{minted}[breaklines=true,frame=single]{python}
class Fruit:
  def __init__(self, name):
    print("-> Initializing")
    self.name = name

orange = Fruit('orange')
print(orange.name)
\end{minted}
\end{frame}


\begin{frame}[containsverbatim]{Fun ones}
  There are some fun ones, such as when objects get added:
  \begin{minted}[fontsize=\small,breaklines=true,frame=single]{python}
class Fruit:
    def __init__(self, name):
        self.type = name
    def __add__(self, other):
        if hasattr(other, 'type'):
            if other.type == 'orange' and self.type == 'apple':
                return 'banana'
        return None

apple = Fruit('apple')
orange = Fruit('orange')

print(apple + orange)
print(orange + apple)
\end{minted}
\end{frame}

\begin{frame}{No Privacy!}
  In Python there is not such thing as a "private" variable as you would in C++.

  Convension is to have "private" variables start with an underscore, and hope nobody accesses it!
\end{frame}

\begin{frame}[containsverbatim]{Well...Sort of}
  There is a not-known feature known as name mangling, where a method with two leading underscores (but no more than 1 trailing one) like \mintinline{python}{__var} will be replaced with \mintinline{python}{_classname__var}.

  You can still access the \mintinline{python}{_classname__var} method.

  So sort of useful I guess, but not really used though.
\end{frame}

\begin{frame}[containsverbatim]{Subclassing}
  Classes can be sub-classes, which the sub-class will inherit all attribues. You can overide them, or re-call them with \verb|super()|

\begin{minted}[fontsize=\scriptsize,breaklines=true,frame=single]{python}
class Fruit:
    def buy(self):
      print("Give Kromer")
    def mix(self):
      print("Mixing")
    def eat(self):
      print("Eating...yummy")

class Watermelon(Fruit):
    def eat(self):
      super().eat()
      #Fruit.eat(self)  # possible, don't use!
      print("\tVery water-y")
    def mix(self):
      print("Not mixable")

w = Watermelon()
w.buy()
w.mix()
w.eat()
\end{minted}
\end{frame}

\begin{frame}[standout]{End}
  The end
\end{frame}

\end{document}
